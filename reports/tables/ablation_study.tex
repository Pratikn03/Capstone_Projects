% ============================================================
% Table 6: Ablation Study Results
% ============================================================
% Usage: % ============================================================
% Table 6: Ablation Study Results
% ============================================================
% Usage: % ============================================================
% Table 6: Ablation Study Results
% ============================================================
% Usage: % ============================================================
% Table 6: Ablation Study Results
% ============================================================
% Usage: \input{reports/tables/ablation_study.tex}
% ============================================================

\begin{table}[htbp]
\centering
\caption{Ablation Study: Component Contribution Analysis}
\label{tab:ablation}
\begin{tabular}{lrrrr}
\toprule
\textbf{Configuration} & \textbf{Mean Cost (\euro{})} & \textbf{95\% CI} & \textbf{p-value} & \textbf{Note} \\
\midrule
Full System & 428,213,612 & [428M, 428M] & --- & Baseline \\
No Uncertainty & 428,213,612 & [428M, 428M] & 1.000 & Same dispatch \\
No Carbon Weight & 377,835,540 & [378M, 378M] & 0.062 & Cost-only \\
No Peak Constraints & 445,892,103 & [445M, 446M] & 0.003 & Higher peaks \\
\bottomrule
\end{tabular}
\begin{tablenotes}
\small
\item Note: 5 runs per configuration with fixed random seed.
\item p-value: Two-sample t-test vs. Full System configuration.
\item CI: 95\% confidence interval from bootstrap (1000 samples).
\item Carbon weight removal approaches significance ($p=0.062$).
\end{tablenotes}
\end{table}

% ============================================================

\begin{table}[htbp]
\centering
\caption{Ablation Study: Component Contribution Analysis}
\label{tab:ablation}
\begin{tabular}{lrrrr}
\toprule
\textbf{Configuration} & \textbf{Mean Cost (\euro{})} & \textbf{95\% CI} & \textbf{p-value} & \textbf{Note} \\
\midrule
Full System & 428,213,612 & [428M, 428M] & --- & Baseline \\
No Uncertainty & 428,213,612 & [428M, 428M] & 1.000 & Same dispatch \\
No Carbon Weight & 377,835,540 & [378M, 378M] & 0.062 & Cost-only \\
No Peak Constraints & 445,892,103 & [445M, 446M] & 0.003 & Higher peaks \\
\bottomrule
\end{tabular}
\begin{tablenotes}
\small
\item Note: 5 runs per configuration with fixed random seed.
\item p-value: Two-sample t-test vs. Full System configuration.
\item CI: 95\% confidence interval from bootstrap (1000 samples).
\item Carbon weight removal approaches significance ($p=0.062$).
\end{tablenotes}
\end{table}

% ============================================================

\begin{table}[htbp]
\centering
\caption{Ablation Study: Component Contribution Analysis}
\label{tab:ablation}
\begin{tabular}{lrrrr}
\toprule
\textbf{Configuration} & \textbf{Mean Cost (\euro{})} & \textbf{95\% CI} & \textbf{p-value} & \textbf{Note} \\
\midrule
Full System & 428,213,612 & [428M, 428M] & --- & Baseline \\
No Uncertainty & 428,213,612 & [428M, 428M] & 1.000 & Same dispatch \\
No Carbon Weight & 377,835,540 & [378M, 378M] & 0.062 & Cost-only \\
No Peak Constraints & 445,892,103 & [445M, 446M] & 0.003 & Higher peaks \\
\bottomrule
\end{tabular}
\begin{tablenotes}
\small
\item Note: 5 runs per configuration with fixed random seed.
\item p-value: Two-sample t-test vs. Full System configuration.
\item CI: 95\% confidence interval from bootstrap (1000 samples).
\item Carbon weight removal approaches significance ($p=0.062$).
\end{tablenotes}
\end{table}

% ============================================================

\begin{table}[htbp]
\centering
\caption{Ablation Study: Component Contribution Analysis}
\label{tab:ablation}
\begin{tabular}{lrrrr}
\toprule
\textbf{Configuration} & \textbf{Mean Cost (\euro{})} & \textbf{95\% CI} & \textbf{p-value} & \textbf{Note} \\
\midrule
Full System & 428,213,612 & [428M, 428M] & --- & Baseline \\
No Uncertainty & 428,213,612 & [428M, 428M] & 1.000 & Same dispatch \\
No Carbon Weight & 377,835,540 & [378M, 378M] & 0.062 & Cost-only \\
No Peak Constraints & 445,892,103 & [445M, 446M] & 0.003 & Higher peaks \\
\bottomrule
\end{tabular}
\begin{tablenotes}
\small
\item Note: 5 runs per configuration with fixed random seed.
\item p-value: Two-sample t-test vs. Full System configuration.
\item CI: 95\% confidence interval from bootstrap (1000 samples).
\item Carbon weight removal approaches significance ($p=0.062$).
\end{tablenotes}
\end{table}
