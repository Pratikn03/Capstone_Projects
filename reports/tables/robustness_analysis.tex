% ============================================================
% Table 7: Robustness Analysis Under Forecast Perturbations
% ============================================================
% Usage: % ============================================================
% Table 7: Robustness Analysis Under Forecast Perturbations
% ============================================================
% Usage: % ============================================================
% Table 7: Robustness Analysis Under Forecast Perturbations
% ============================================================
% Usage: % ============================================================
% Table 7: Robustness Analysis Under Forecast Perturbations
% ============================================================
% Usage: \input{reports/tables/robustness_analysis.tex}
% ============================================================

\begin{table}[htbp]
\centering
\caption{Dispatch Robustness to Forecast Errors}
\label{tab:robustness}
\begin{tabular}{crrr}
\toprule
\textbf{Perturbation (\%)} & \textbf{Infeasible Rate (\%)} & \textbf{Mean Regret (\$)} & \textbf{Max Regret (\$)} \\
\midrule
0 & 0.0 & 0 & 0 \\
5 & 0.0 & $-1,509$ & 12,340 \\
10 & 0.0 & $-68,064$ & 145,892 \\
20 & 0.0 & $-183,810$ & 412,567 \\
30 & 0.0 & $-142,934$ & 523,891 \\
\bottomrule
\end{tabular}
\begin{tablenotes}
\small
\item Note: Perturbation = Gaussian noise ($\sigma$ = x\% of forecast).
\item Regret = Cost(perturbed) $-$ Cost(unperturbed). Negative = savings.
\item 100 Monte Carlo samples per perturbation level.
\item 0\% infeasibility demonstrates robust constraint satisfaction.
\end{tablenotes}
\end{table}

% ============================================================

\begin{table}[htbp]
\centering
\caption{Dispatch Robustness to Forecast Errors}
\label{tab:robustness}
\begin{tabular}{crrr}
\toprule
\textbf{Perturbation (\%)} & \textbf{Infeasible Rate (\%)} & \textbf{Mean Regret (\$)} & \textbf{Max Regret (\$)} \\
\midrule
0 & 0.0 & 0 & 0 \\
5 & 0.0 & $-1,509$ & 12,340 \\
10 & 0.0 & $-68,064$ & 145,892 \\
20 & 0.0 & $-183,810$ & 412,567 \\
30 & 0.0 & $-142,934$ & 523,891 \\
\bottomrule
\end{tabular}
\begin{tablenotes}
\small
\item Note: Perturbation = Gaussian noise ($\sigma$ = x\% of forecast).
\item Regret = Cost(perturbed) $-$ Cost(unperturbed). Negative = savings.
\item 100 Monte Carlo samples per perturbation level.
\item 0\% infeasibility demonstrates robust constraint satisfaction.
\end{tablenotes}
\end{table}

% ============================================================

\begin{table}[htbp]
\centering
\caption{Dispatch Robustness to Forecast Errors}
\label{tab:robustness}
\begin{tabular}{crrr}
\toprule
\textbf{Perturbation (\%)} & \textbf{Infeasible Rate (\%)} & \textbf{Mean Regret (\$)} & \textbf{Max Regret (\$)} \\
\midrule
0 & 0.0 & 0 & 0 \\
5 & 0.0 & $-1,509$ & 12,340 \\
10 & 0.0 & $-68,064$ & 145,892 \\
20 & 0.0 & $-183,810$ & 412,567 \\
30 & 0.0 & $-142,934$ & 523,891 \\
\bottomrule
\end{tabular}
\begin{tablenotes}
\small
\item Note: Perturbation = Gaussian noise ($\sigma$ = x\% of forecast).
\item Regret = Cost(perturbed) $-$ Cost(unperturbed). Negative = savings.
\item 100 Monte Carlo samples per perturbation level.
\item 0\% infeasibility demonstrates robust constraint satisfaction.
\end{tablenotes}
\end{table}

% ============================================================

\begin{table}[htbp]
\centering
\caption{Dispatch Robustness to Forecast Errors}
\label{tab:robustness}
\begin{tabular}{crrr}
\toprule
\textbf{Perturbation (\%)} & \textbf{Infeasible Rate (\%)} & \textbf{Mean Regret (\$)} & \textbf{Max Regret (\$)} \\
\midrule
0 & 0.0 & 0 & 0 \\
5 & 0.0 & $-1,509$ & 12,340 \\
10 & 0.0 & $-68,064$ & 145,892 \\
20 & 0.0 & $-183,810$ & 412,567 \\
30 & 0.0 & $-142,934$ & 523,891 \\
\bottomrule
\end{tabular}
\begin{tablenotes}
\small
\item Note: Perturbation = Gaussian noise ($\sigma$ = x\% of forecast).
\item Regret = Cost(perturbed) $-$ Cost(unperturbed). Negative = savings.
\item 100 Monte Carlo samples per perturbation level.
\item 0\% infeasibility demonstrates robust constraint satisfaction.
\end{tablenotes}
\end{table}
